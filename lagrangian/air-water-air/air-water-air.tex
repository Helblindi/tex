\documentclass{article}

\usepackage{amsmath, amsthm, amssymb, amsfonts}
\usepackage{graphicx}
%It is best practice to keep all your pictures in
%one folder inside the main directory in which your
%TeX file is kept. Here the folder is named "images."
%Replace the name here with your folder's name, if needed.
%The period is needed due to relative referencing.
\graphicspath{ {./images/} }
%If you have JPEG format images, add .jpg as an
%allowed file extension below. Same for Bitmaps (.bmp).
\DeclareGraphicsExtensions{.png}
\usepackage{setspace}
\usepackage{geometry}
\usepackage{float}
\usepackage[hidelinks]{hyperref}
\usepackage[utf8]{inputenc}
\usepackage[english]{babel}
\usepackage{framed}
\usepackage[dvipsnames]{xcolor}
\usepackage{tcolorbox}

\usepackage{color}
\usepackage{bbm}
\usepackage{dsfont}
\usepackage{xspace}


% For natbib-style references, uncomment this.
\usepackage[numbers]{natbib}
% \let\cite=\citet
% Use \citet for textual citations Author [#]
% Use \cite for normal citations
% Use \citep for optional pre/post notes
% Ex: \citep[see][p.~5]{Smith2020} gives (see Smith, [#], p. 5).

\usepackage{array}
\usepackage{enumerate}
\usepackage{enumitem}
\usepackage{ulem}
% \let\cite=\citet
\usepackage{adjustbox}
\usepackage{caption}
\usepackage{subcaption}
\usepackage{siunitx}
\usepackage{algorithm}
\usepackage{algpseudocode}

% Custom definitions
\makeatletter
%EDIT the path to the .sty files as needed
\def\input@path{{../../}} % Adjust relative to the current file
\makeatother
\usepackage{mydef}
\usepackage{xthm}
\usepackage{comments}

% -------------- Project specific commands
\newcommand{\HRule}[1]{\rule{\linewidth}{#1}}
% --------------

% \setstretch{1.2}
\geometry{
    textheight=9in,
    textwidth=5.5in,
    top=1in,
    headheight=12pt,
    headsep=25pt,
    footskip=30pt
}

% ------------------------------------------------------------------------------

\begin{document}

% ------------------------------------------------------------------------------
% Cover Page and ToC
% ------------------------------------------------------------------------------

\title{ \normalsize \textsc{}
		\\ [2.0cm]
		\HRule{1.5pt} \\
		\LARGE \textbf{\uppercase{Air Water Air test case}
		\HRule{2.0pt} \\ [0.6cm] 
      % \LARGE{Subtitle}
      \vspace*{10\baselineskip}}
		}
\date{}
\author{\textbf{Author} \\ 
		Madison Sheridan}

\maketitle
\newpage

\tableofcontents
\newpage

% ------------------------------------------------------------------------------

\section{Problem Description}
We describe the air-water-air test that was introduced in \cite{shyue-2004}, and lated tested in \cite{cheng-shu-2014, Vilar_Shu_Maire_2D}.
%
In this test, we consider a radially symmetric two-phase configuration consisting of an inner air region, a middle water layer, and an outer air region. The computational domain corresponds to a cylinder of radius $r \in [0,1.2]$, where the material distribution is defined as follows:
\begin{equation}
(\rho, \bv, p) =
\begin{cases}
(0.001, \bzero, 1000), & 0 < r < 0.2, \\[4pt]
(1, \bzero, 1), & 0.2 < r < 1.0, \\[4pt]
(0.001, \bzero, 0.001), & 1.0 < r < 1.2.
\end{cases}
\label{eq:air-water-air-ic}
\end{equation}
The first and last regions correspond to air and are modeled by the ideal gas EOS with $\gamma = 1.4$.
The middle region contained water and is modeled by the Stiffened Gas EOS with $\gamma = 7$ and $p_s = 3000$.
The problem thus models the interaction between compressible air and nearly incompressible water layers under Lagrangian motion.

Because of the radial symmetry of the configuration, we restrict computations to a quarter of the domain and apply symmetry boundary conditions along the bottom, left, and right boundaries.

This test is designed to assess the robustness and positivity-preserving properties of the Lagrangian scheme in the presence of large material density contrasts. In particular, it is known that without a positivity-preserving limiter, the internal energy may become negative in the air regions, leading to numerical instability. Therefore, this case provides a stringent test for evaluating the ability of the limiter to maintain the physical admissibility of density and internal energy throughout the evolution.

% ------------------------------------------------------------------------------

\newpage

% ------------------------------------------------------------------------------
% Reference and Cited Works
% ------------------------------------------------------------------------------
%fix spacing in bibliography, if any...
%%%%%%%%%%%%%%%%%%%%%%%%%%%%%%%%%%%%%%%%%%%%%%%%%%%%%%%%%%%%%
\let\oldbibitem\bibitem
\renewcommand{\bibitem}{\setlength{\itemsep}{0pt}\oldbibitem}
%%%%%%%%%%%%%%%%%%%%%%%%%%%%%%%%%%%%%%%%%%%%%%%%%%%%%%%%%%%%%%%
%The bibliography style declared is the IEEE format. If
%you require a different style, see the document
%bibstyles.pdf included in this package. This file,
%hosted by the University of Vienna, shows several
%bibliography styles and examples of in-text citation
%and a references page.
% \bibliographystyle{ieeetr}
\bibliographystyle{plainnat}

% \phantomsection
% \addcontentsline{toc}{chapter}{REFERENCES}

% \renewcommand{\bibname}{{\normalsize\rm REFERENCES}}

%This file is a .bib database that contains the sources.
%This removes the dependency on the previous file
%bibliography.tex.
%EDIT the path to the .bib file as needed
\bibliography{../../bibliography}

% ------------------------------------------------------------------------------

\end{document}
