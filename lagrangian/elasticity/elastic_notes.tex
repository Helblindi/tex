\documentclass{article}

\usepackage{amsmath, amsthm, amssymb, amsfonts}
\usepackage{graphicx}
%It is best practice to keep all your pictures in
%one folder inside the main directory in which your
%TeX file is kept. Here the folder is named "images."
%Replace the name here with your folder's name, if needed.
%The period is needed due to relative referencing.
\graphicspath{ {./images/} }
%If you have JPEG format images, add .jpg as an
%allowed file extension below. Same for Bitmaps (.bmp).
\DeclareGraphicsExtensions{.png}
\usepackage{setspace}
\usepackage{geometry}
\usepackage{float}
\usepackage[hidelinks]{hyperref}
\usepackage[utf8]{inputenc}
\usepackage[english]{babel}
\usepackage{framed}
\usepackage[dvipsnames]{xcolor}
\usepackage{tcolorbox}

\usepackage{color}
\usepackage{bbm}
\usepackage{dsfont}
\usepackage{xspace}


% For natbib-style references, uncomment this.
\usepackage[numbers]{natbib}
% \let\cite=\citet
% Use \citet for textual citations Author [#]
% Use \cite for normal citations
% Use \citep for optional pre/post notes
% Ex: \citep[see][p.~5]{Smith2020} gives (see Smith, [#], p. 5).

\usepackage{array}
\usepackage{enumerate}
\usepackage{enumitem}
\usepackage{ulem}
% \let\cite=\citet
\usepackage{adjustbox}
\usepackage{caption}
\usepackage{subcaption}
\usepackage{siunitx}
\usepackage{algorithm}
\usepackage{algpseudocode}

% Custom definitions
\makeatletter
%EDIT the path to the .sty files as needed
\def\input@path{{../../}} % Adjust relative to the current file
\makeatother
\usepackage{mydef}
\usepackage{xthm}
\usepackage{comments}

% -------------- Project specific commands
\newcommand{\HRule}[1]{\rule{\linewidth}{#1}}
% --------------

% \setstretch{1.2}
\geometry{
    textheight=9in,
    textwidth=5.5in,
    top=1in,
    headheight=12pt,
    headsep=25pt,
    footskip=30pt
}

% ------------------------------------------------------------------------------

\begin{document}

% ------------------------------------------------------------------------------
% Cover Page and ToC
% ------------------------------------------------------------------------------

\title{ \normalsize \textsc{}
		\\ [2.0cm]
		\HRule{1.5pt} \\
		\LARGE \textbf{\uppercase{Elastic Notes}
		\HRule{2.0pt} \\ [0.6cm] 
      % \LARGE{Subtitle}
      \vspace*{10\baselineskip}}
		}
\date{}
\author{\textbf{Author} \\ 
		Madison Sheridan}

\maketitle
\newpage

\tableofcontents
\newpage

% ------------------------------------------------------------------------------

\section{Anisotropic Derivation}

\subsection{Notation}
Recall the deformation gradient
\begin{equation}
   \polJ := \frac{\partial \bPhi\left(\bxi, t\right)}{\partial \bxi}
\end{equation}
where $\bPhi$ is the mapping from reference to current configuration.
The right Cauchy-Green deformation tensor is defined as
\begin{equation}
   \bC := \polJ\tr\polJ.
\end{equation}
The reduced right Cauchy-Green deformation tensor is defined as
\begin{equation}
   \obc := \frac{\bC}{\left|\bC\right|^{1/3}}
\end{equation}

\subsection{Hyperelasticity}
In order to guarantee the existence of solutions, the strain energy function must be sequentially weakly lower semicontinuous (s.w.l.s.) and must meet a coercivity condition \cite{schroder-neff-ebbing-2008}.

\MS{Define s.w.l.s. and coercivity condition.}

Adopting Ball’s polyconvexity framework \cite{ball-1977} guarantees sequential weak lower semicontinuity and coercivity and, in particular, implies ellipticity. For this reason, we formulate the strain-energy function to be polyconvex.

\begin{definition}[Polyconvexity]
   A function $W\in C^2\left(\polM^{3\times 3}, \polR\right)$ is polyconvex if there exists a convex function $P:\mathbb{R}^{3\times 3}\times\mathbb{R}^{3\times 3}\times\mathbb{R}\to\mathbb{R}$ such that
   \begin{equation}
      W(\polJ) = P(\polJ, \text{Adj}\,\polJ, \det\,\polJ)
   \end{equation}
   for all $\polJ\in\mathbb{R}^{3\times 3}$ with $\det\,\polJ > 0$.
\end{definition}

\subsection{Uniaxial Tension Test}

In \cite[Sec.~4.1]{chaimoon2019} for the Uniaxial tension tests on coronary arteries the authors describe their fiber based stress to account for the anisotropic case.
They define the strain energy function in Eq.~(39) as
\begin{multline}
   \label{eqn:chaimoon-strain}
   \bsigma = \left(1-2w_1^1\right) \left[\frac{A_1}{2}\left(I_1-3\right) + \frac{B_1}{2}\left(\frac{I_2}{I_3}-3\right)\right] \\
   + 2w_1^1D_1 \left[\frac{1}{A_1}\left(e^{A_1\left(J_{4i}-1\right)}-1\right) + \frac{1}{B_1}\left(e^{B_1\left(K_{5i}^{\text{inc}}-1\right)} -1\right)\right]
\end{multline}
where
\begin{align*}
   I_1 &= \Tr\left[\bC\right] & J_{4i} &= \Tr\left[\bC\bG_i\right] \\
   I_2 &= \frac{1}{2}\left(\Tr\left[\bC\right]^2 - \Tr\left[\bC^2\right]\right) & J_{5i} &= \Tr\left[\bC^2\bG_i\right] \\
   I_3 &= \det\left[\bC\right] & K_{5i} &= J_{5i} - I_1J_{4i} + I_2\Tr\left[\bG_i\right].
\end{align*}

\begin{remark}
   Notice that in \cite{chaimoon2019} they work in the Cauchy-Green Strain tensor $\bC$, not the reduced tensor $\obc$.
   This will mean slight modifications to our representation.
\end{remark}

\begin{remark}
   Notice also that the strain energy function given in \eqref{eqn:chaimoon-strain} is in terms of $K_{5i}$. 
   This is because the invariant $J_{5i}$ is not convex with respect to $\polJ$, as shown in \cite[Lem.~C.7]{schroder-neff-2003}.
\end{remark}

\subsection{Reduced Invariant Implementation}
We adopt the following reduced invariants
\begin{align*}
   j_1 &= \Tr\left[\obc\right] & j_{4i} &= \Tr\left[\obc\bG_i\right] \\
   j_2 &= \Tr\left[\obc^2\right] & j_{5i} &= \Tr\left[\obc^2\bG_i\right] \\
   j_3 &= \det\left[\obc\right] = 1 & k_{5i} &= j_{5i} - j_1j_{4i} + j_2\Tr\left[\bG_i\right].   
\end{align*}
Then we have the following relationships between the invariants:
\begin{align*}
   I_1 &= \left|\bC\right|^{1/3} j_1 & J_{4i} &= \left|\bC\right|^{1/3} j_{4i} \\
   I_2 &= \left|\bC\right|^{2/3}\frac{j_1^2 - j_2}{2} & J_{5i} &= \left|\bC\right|^{2/3} j_{5i} \\
   & & K_{5i} &= \left|\bC\right|^{2/3}\left(j_{5i} - j_1j_{4i} + \frac{j_1^2 - j_2}{2}\right).
\end{align*}
Then the strain energy function \eqref{eqn:chaimoon-strain} becomes 
\begin{multline}
\label{eqn:aortic-eos}
    e^s=\left(1-2 w^1_1 \right)\left[\frac{A_1}{2}\left(\left|\bC\right|^{1/3}j_1-3\right)+\frac{B_1}{2}\left(\frac{1}{2\left|\bC\right|^{1/3}}\left(j_1^2-j_2\right)-3\right)\right]\\
    +2w^1_1D_1\left[\frac{1}{A_1}\left(\exp\left\{A_1\left(\left|\bC\right|^{1/3}j_{4i}-1\right)\right\}-1\right)+\frac{1}{B_1}\left(\exp\left\{\left(B_1\left(\left|\bC\right|^{2/3} \left(j_{5i}-j_1j_{4}+\frac{j_1^2-j_2}{2}\right)-1\right)\right)\right\}-1\right)\right]
\end{multline}

\newpage

% ------------------------------------------------------------------------------
% Reference and Cited Works
% ------------------------------------------------------------------------------
%fix spacing in bibliography, if any...
%%%%%%%%%%%%%%%%%%%%%%%%%%%%%%%%%%%%%%%%%%%%%%%%%%%%%%%%%%%%%
\let\oldbibitem\bibitem
\renewcommand{\bibitem}{\setlength{\itemsep}{0pt}\oldbibitem}
%%%%%%%%%%%%%%%%%%%%%%%%%%%%%%%%%%%%%%%%%%%%%%%%%%%%%%%%%%%%%%%
%The bibliography style declared is the IEEE format. If
%you require a different style, see the document
%bibstyles.pdf included in this package. This file,
%hosted by the University of Vienna, shows several
%bibliography styles and examples of in-text citation
%and a references page.
% \bibliographystyle{ieeetr}
\bibliographystyle{plainnat}

% \phantomsection
% \addcontentsline{toc}{chapter}{REFERENCES}

% \renewcommand{\bibname}{{\normalsize\rm REFERENCES}}

%This file is a .bib database that contains the sources.
%This removes the dependency on the previous file
%bibliography.tex.
%EDIT the path to the .bib file as needed
\bibliography{../../bibliography}

% ------------------------------------------------------------------------------

\end{document}
